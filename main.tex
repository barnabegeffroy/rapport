\documentclass[a4paper]{report}

%====================== PACKAGES ======================

\usepackage{chngcntr}
\counterwithout{footnote}{chapter}

\usepackage{fontawesome}

\usepackage{xcolor}
\usepackage{listings}
\lstset{basicstyle=\ttfamily,
  showstringspaces=false,
  commentstyle=\color{red},
  keywordstyle=\color{blue},
  numbers=left,
  stepnumber=1,
  showstringspaces=false,
  tabsize=1,
  breaklines=true,
  breakatwhitespace=false
}

\usepackage{authblk}
\usepackage{caption}
\usepackage[french]{babel}
\usepackage[utf8x]{inputenc}
%pour gérer les positionnement d'images
\usepackage{float}
\usepackage{amsmath}
\usepackage{graphicx}
\usepackage[colorinlistoftodos]{todonotes}
\usepackage{url}
%pour les informations sur un document compilé en PDF et les liens externes / internes
\usepackage{hyperref}
%pour la mise en page des tableaux
\usepackage{array}
\usepackage{tabularx}
%pour utiliser \floatbarrier
%\usepackage{placeins}
%\usepackage{floatrow}
%espacement entre les lignes
\usepackage{setspace}
%modifier la mise en page de l'abstract
\usepackage{abstract}
%police et mise en page (marges) du document
\usepackage[T1]{fontenc}
\usepackage[top=2cm, bottom=2cm, left=2cm, right=2cm]{geometry}
%Pour les galerie d'images
\usepackage{subfig}

%====================== INFORMATION ET REGLES ======================

\newcommand{\source}[1]{\caption*{\small{\textsc{Source:} {#1}}} }

\addto\captionsfrench{%
  \renewcommand\chaptername{Projet}}
%rajouter les numérotation pour les \paragraphe et \subparagraphe
\setcounter{secnumdepth}{4}
\setcounter{tocdepth}{1}

\hypersetup{							% Information sur le document
pdfauthor = {Barnabé Geffroy},			% Auteurs
pdftitle = {Rapport de stage},			% Titre du document
pdfsubject = {Rapport de stage},		% Sujet
pdfstartview={FitH}}					% ajuste la page à la largueur de l'écran
%pdfcreator = {MikTeX},% Logiciel qui a crée le document
%pdfproducer = {}} % Société avec produit le logiciel

%======================== DEBUT DU DOCUMENT ========================

\begin{document}

%régler l'espacement entre les lignes
\newcommand{\HRule}{\rule{\linewidth}{0.5mm}}

%page de garde
\begin{titlepage}
\begin{center}

% Upper part of the page. The '~' is needed because only works if a paragraph has started.
\includegraphics[width=0.35\textwidth]{assets/logo.jpg}~\\[1cm]

\textsc{\LARGE CPES 3}\\[1.5cm]

\textsc{\Large }\\[0.5cm]

% Title
\HRule \\[0.4cm]

{\huge \bfseries Rapport de stage\\
Développement agile d’applications web \\[0.4cm] }

\HRule \\[1.5cm]

% Author and supervisor
\begin{minipage}{0.4\textwidth}
\begin{flushleft} \large
\emph{Auteur:}\\
Barnabé \textsc{Geffroy}\\
\end{flushleft}
\end{minipage}
\begin{minipage}{0.4\textwidth}
\begin{flushright} \large
\emph{Référent:} \\
Olivier \textsc{Cailloux}
\end{flushright}
\end{minipage}

\vfill

% Bottom of the page
{\large \today}

\end{center}
\end{titlepage}



\tableofcontents
\thispagestyle{empty}
\setcounter{page}{0}
%ne pas numéroter le sommaire

\setlength\parindent{0pt}
\chapter*{Introduction}

Le développement agile est une méthodologie qui vise à délivrer rapidement une solution fonctionnelle et d'améliorer progressivement le code. Cette méthode permet une simplification du code et d'explorer des voies suggérées lors du développement et inenvisagées initiallement. L'ajout de tests fréquents permet de ne pas se perdre dès que le code renvoie une erreur. Une communication régulière entre les développeurs est également nécessaire en développement agile.

Lors de ce stage, plusieurs projets ont été réalisés en suivant cette méthode. Il seront exposés dans ce rapport. Premièrement, quelques détails techniques seront exposés. Ensuite, les quatre projets seront présentés. Le premier projet traite du développement d'une application web gérant la présence des élèves au sein d'un master de l'Université Paris-Dauphine. Le second projet s'intéresse à une application générant un fichier PDF contenant le programme du même master. Le troisième projet est un projet découlant du second. Celui-ci n'était initiallement pas prévu mais le méthodologie agile nous a conduit à développer une système d'authentification. Le dernier projet est la mise en pratique de l'article \textit{A formal framework for deliberated judgment}\cite{cailloux_formal_2020}. Une annexe à la fin du document propose certains codes utilisés dans les projets. Tous les codes sont également disponibles sur mon profil GitHub : \url{https://github.com/barnabegeffroy}.

%espacement entre les lignes d'un tableau
\renewcommand{\arraystretch}{1.5}

\setlength\parindent{0pt}
%====================== INCLUSION DES PARTIES ======================

~
\thispagestyle{empty}
%recommencer la numérotation des pages à "1"
\setcounter{page}{0}
\newpage

\chapter*{Préliminaires}
Lors de ce stage \texttt{Git} et Maven ont été utilisés sur la plupart des projets. En voici une présentation succinte.
\addcontentsline{toc}{chapter}{Préliminaires}
\section*{\texttt{Git}}
\subsection*{Fonctionnement}
\texttt{Git} est un système de contrôle de version qui permet la collaboration entre développeurs. Le code source est conservé dans un \textit{dépôt} distant. Il suit un modèle distribué, il n'y a pas de serveur central. Le code est donc accessible par plusieurs sources et peut être utilisé sans connexion. La connexion internet est cependant nécessaire pour envoyer ces modifications sur le dépôt distant. Ce genre de sauvegarde est appelé \textit{commit}. Celle-ci est une version du code à instant donné. \texttt{Git} crée, avec tous les commits, une série d’instantanés qui rend la perte d'information très difficile. 

\texttt{Git} gère également l'intégrité du code. Il peut y avoir des conflits, des parties identiques du code modifiées par deux utlisateurs. \texttt{Git} pointe les régions du code qui sont différentes et les utlisateurs éditent le code pour régler les zones de conflit.

\subsection*{Quelques commandes}

\begin{itemize}
    \item \texttt{git init} initialise un nouveau dépôt.
    \item \texttt{git clone} copie un dépôt \texttt{Git} déjà existant.
    \item \texttt{git add} ajoute les fichiers que l'on veut sauvegarder dans le commit.
    \item \texttt{git status} affiche l'état des fichiers (ajouté ou non).
    \item \texttt{git commit} crée un instantané du code en modifiant les fichiers ajoutés avec \texttt{git add}.
    \item \texttt{git push} envoie les commits sur le dépôt distant.
\end{itemize}

Seules les commandes \texttt{git clone} et  \texttt{git push} nécessitent une connexion internet. Il est donc très aisé de travailler 

\subsubsection*{GitHub \href{https://github.com/barnabegeffroy}{\faGithub}}
GitHub est un hébergeur de dépôts \texttt{Git}. Il offre la gestion de version distribuée, la fonctionnalité de gestion de code source de \texttt{Git}, ainsi que ses propres fonctionnalités. Il compte plus de 50 millions d'inscrits et plus de 100 millions de dépôts.

\section*{Maven}
Maven est un outil de gestion de configuration de projet, en particulier de gestion des dépendances. Il permet de ne pas se soucier de l’environnement de compilation. Les dépendances sont indiquées dans un fichier nommé \texttt{pom.xml}. Grâce à ce fichier Maven configure les bibliothèque et autres dépendances. Maven propose aussi une structure du projet qui sera la même dans chacun des projets exposés. 

Voici l'arborescence de base d'un projet Maven :
\begin{center}
    \begin{minipage}[t]{3.5cm}
        \texttt{pom.xml}\\
        \texttt{/src}
        \begin{itemize}
            \item[] \texttt{/main}
                  \begin{itemize}
                      \item[] \texttt{/java}
                  \end{itemize}\vspace{-0.8ex}
                  \begin{itemize}
                      \item[] \texttt{/resources}
                  \end{itemize}\vspace{-0.8ex}
            \item[] \texttt{/test}
                  \begin{itemize}
                      \item[] \texttt{/java}
                  \end{itemize}\vspace{-0.8ex}
                  \begin{itemize}
                      \item[] \texttt{/resources}
                  \end{itemize}\vspace{-0.8ex}
        \end{itemize}
    \end{minipage}\hfill%
\end{center}

\chapter[Gestion des présences]{Gestion des présences\raisebox{.3\baselineskip}{\normalsize\footnotemark}}
\footnotetext{\url{https://github.com/barnabegeffroy/Attendance}}

L'offre de stage\footnote{\href{https://github.com/Dauphine-MIDO/M1-app/blob/master/Stage dev.adoc}{\textcolor{blue}{\underline{Offre de stage: développement agile d’applications web et de bibliothèques open-source}}}} auquel je me suis porté candidat portait initialement sur le développement d'une application gérant la présence des élèves du Master 1 MIAGE en Apprentissage. L'idée de ce projet était de digitaliser les feuilles de présences.

\section{Attendance}

Le projet Attendance avait donc pour but de développer une application web liée à la gestion des présences des élèves. Le serveur HTTP Eclipse Jetty, 

\begin{figure}[!h]
    \begin{center}
    %taille de l'image en largeur
    %remplacer "width" par "height" pour régler la hauteur
    \includegraphics[width=5cm]{assets/attendance.PNG}
    \end{center}
    %légende de l'image
    \caption{Enveloppe d'un son}
\end{figure}

\section{JeSuisEnCours}

Après quelques semaines, nous nous sommes rendus compte que la direction du projet était incompatible avec les exigences administratives. En effet, l'application prévoyait un simple appel du professeur, or l'étudiant doit personnellement attester de sa présence en émargeant un document. Il donc été décidé d'abandonner le projet intiale Attendance pour se tourner vers une application JeSuisEnCours, spécialisée dans la digitalisation les feuilles de présences. 

Le but principal du projet est donc devenu la connexion de l'application JeSuisEnCours aux données de l'université, d'une part, pour accéder aux données (annuaires, emploies du temps,...), d'autre part pour gérer les éléments renvoyés par l'application (absence, justificatif,...).

Malheureursement, la crise sanitaire a fortement ralentit les contacts avec l'équipe de JeSuisEnCours et le projet a finalement été abandonné.
 
\chapter[Plaquette-MIDO]{Plaquette-MIDO\raisebox{.3\baselineskip}{\normalsize\footnotemark}}
\footnotetext{\url{https://github.com/Dauphine-MIDO/plaquette-MIDO}}

Ce projet a pour but d'implémenter un code générant un fichier PDF détaillant les différents enseignements du Master 1 MIAGE en apprentissage à partir de la base de données de Dauphine. La finalité est d'automatiser le lancement du code de sorte que quotidiennement le fichier PDF soit mis à jour et publié en ligne.

À mon arrivée, un code permettait déjà la génération du fichier PDF. Certains passages du code étaient cependant à revoir pour améliorer l'esthétique du fichier PDF. De plus, le code initial était très peu généralisé à d'autres utilisateurs et plusieurs changements étaient nécessaire pour qu'un autre utilisateur puisse lancer Plaquette-MIDO. Le principal aspect à généraliser était l'authentification à l'API de Dauphine\footnote{Une API est une interface de programmation d’application qui permet d'accéder à un ensemble de classes, méthodes, fonctions et autres données. Dans le cas de Dauphine, son API donne accès aux fonctions informatiques permettant de manipuler le programme des cours des différentes formations}, indispensable pour avoir accès aux données de l'université.

\section{L'automatisation du lancement du code}
\label{sec:automatisation}

La finalité du projet Plaquette-MIDO est de lancer la construction du fichier PDF quotidiennement et de le publier de manière à ce qu'il soit accessible sur le site de Dauphine. Pour réaliser cette automatisation, j'ai utilisé l'outil Travis-CI.

\subsection{Travis-CI}
    Travis-CI est un logiciel d'intégration continue qui permet de compiler,
    tester et déployer le code de dépôts GitHub. Il est configuré à partir
    d'un fichier nommé \texttt{.travis.yml} présent dans la racine du
    répertoire. Celui-ci est lu à chaque nouveau commit par Travis-CI qui
    exécute son contenu sur une machine virtuelle. L'exécution peut alors
    réussir dans le cas où aucune erreur n'a été signalée ou échouer si une
    erreur est survenue. Travis-CI peut ainsi s'assurer de la bonne
    compilation d'un projet. Il peut également effectuer des déploiements.
    En effet, il est possible d'insérer du script que Travis-CI exécute pour
    déployer des fichiers. Par exemple, ajouter un script avec les commandes
    git adéquat pour pousser des fichiers vers un dépôt. Travis-CI exécute le code source permettant la création du fichier PDF et ensuite
    déploie ce fichier vers un dépôt GitHub.

\begin{figure}[!ht]
    \begin{center}
        \includegraphics[width=11cm]{assets/travis-ci.PNG}
    \end{center}
    \caption{Capture d'écran de l'interface de Travis-CI}
    \label{travis}
\end{figure}

La figure~\ref{travis} montre que la construction du commit \texttt{just push logs} a échoué pour le dépôt \texttt{plaquette}, tandis que celle du dépôt \texttt{vegan\_or\_not} a réussi.

Travis-CI permet également de référencer des variables d'environnement sécurisées (clefs d'accès, identifiants, ...).
\begin{figure}[!ht]
    \begin{center}
    \includegraphics[width=0.8\textwidth]{assets/env.PNG}
    \caption{Différentes variables d'environnement entréees pour la construction Travis-CI d'un dépôt}
    \label{env}
    \end{center}

\end{figure}

\subsection{L'exécution du code}
\subsubsection*{Les dépendances}



Travis-CI identifie automatiquement un projet Maven et installe les dépendances indiquées dans le \texttt{pom.xml}. Dans le projet Plaquette-MIDO, la dépendance qui importe le code source de l'API de Dauphine nécessite un fichier texte nommé \texttt{WSDL\_Login.txt} contenant un URL spécifique aux identifiants de l'utilisateur. Ce fichier ne peut pas être dans le dépôt car il contient des informations personnelles. Il faut un script qui crée le fichier sur la machine virtuelle de Travis-CI. Le fichier doit être créé avant de lancer l'installation des dépendances . Dans le \texttt{.travis.yml}, on peut ajouter un script qui génère ce fichier. 

Voici ci-dessous un extrait du script permettant la création d'un tel fichier. On y retrouve les variables d'environnement présentées dans la figure~\ref{env}.

\begin{figure}[!ht]
    \lstinputlisting[language=bash, firstline=18]{./assets/writeWSDL}
    \caption*{Extrait de writeWSDL.sh, annexe~\ref{sec:writeWSDL} }
\end{figure}

\subsubsection*{Création du fichier PDF et déploiement}
Une fois que toutes les dépendances du projet Maven sont installées, il faut executer le code source qui génère le fichier PDF. La classe qui permet cette génération est \texttt{M1ApprBuilder}. Le script de Travis-CI va exécuter cette classe, il faudra ensuite déployer vers un dépôt d'arrivée le fichier PDF ainsi que le logs de la construction. La construction de Travis-CI doit échouer si le fichier PDF n'est pas généré. Néanmoins, dans tous les cas les logs doivent être poussés vers le dépôt. Ainsi, si le fichier PDF est généré, il est poussé avec les logs vers le dépôt d'arrivée. Si ce n'est pas le cas, la construction échoue, le dernier PDF déployé reste disponible. Nous sommmes avertis immédiatement par e-mail de cet échec. Les logs du code source de Plaquette-MIDO ainsi que ceux de Travis-CI permettront alors d'expliquer de l'échec.

Vous trouverez le script exécuté par Travis-CI dans l'annexe~\ref{sec:cibuild}. Le script exécute plusieurs fonctions :

\begin{itemize}
    \item \texttt{clean} (l.20 à 24), supprime les fichiers déjà existants.
    \item \texttt{get\_current\_deploy} (l.26 à 30), copie sur la machine virtuelle de Travis-CI le dépôt dans lequel les fichiers vont être déployés. Pour éviter de cloner un dépôt \texttt{Git} dans un autre dépôt \texttt{Git}, celui-ci est cloné dans le dossier parent de Plaquette-MIDO de la machine virtuelle \texttt{"../\${REPO}"}.
    \item \texttt{build\_doc} (l.31 à 41), essaie de générer le document en exécutant \texttt{M1ApprBuilder}. Elle met aussi à jour la variable \texttt{BUILT\_EXIT\_CODE} qui prend la valeur 0 si le code est correctement exécuté et que le fichier PDF est généré, ou la valeur 1 sinon.
    \item \texttt{deploy} (l.43 à 60), déplace les logs et le fichier PDF (s'il y en a un) vers le dépôt d'arrivée (\texttt{"../\${REPO}"}). Les différentes commandes \texttt{Git} présentées ultérieurement sont alors executées pour déployer les fichiers déplacés.\\ La dernière ligne \texttt{exit \${BUILT\_EXIT\_CODE}} renvoie à Travis-CI la valeur mis à jour dans \texttt{build\_doc}. Si la valeur est 0, la construction contiue et s'achève par un succès. Sinon la construction échoue et une notification est envoyée pour prévenir les développeurs.
\end{itemize}


\subsubsection*{Automatisation}
Travis-CI lance initiallement la construction du code du dépôt à chaque nouveau commit. Il est cependant possible de configurer des \textit{Cron Jobs}. Ceux-ci vont renouveler la construction du code de manière quotidienne, hebdomadaire ou mensuelle. 

\begin{figure}[!ht]
    \begin{center}\includegraphics[width=\textwidth]{assets/CronJobs.PNG}
    \end{center}
    \caption{Les \textit{Cron Jobs} de Travis-CI, ici la construction est programmée quotidiennement}
\end{figure}

En programmant sur \textit{Daily}, Travis-CI va relancer la construction du code tous les jours et ainsi renouveller le fichier PDF et les logs si des changements sont effectués. Si une erreur se produit et que le fichier PDF n'est pas généré, une e-mail nous préviendra de la non-construction du code et le fichier PDF le plus récent sera toujours disponible sur le site.

\section{L'authentification}

Pour se connecter à l'API de Dauphine, un nom d'utilisateur et un mot de passe sont nécessaires. Le code initial prévoyait trois manières de fournir ces informations afin de se connecter à l'API :
\begin{itemize}
    \item les propriétés du système
    \item les variables d'environnement
    \item un fichier texte contenant les informations nécessaires
\end{itemize}

Cependant, ce code ne lisait initialement que le mot de passe et le nom d'utilisateur était une valeur par défaut. Un nouvel utilisateur devait modifier le code pour pouvoir utiliser ses identifiants. L'idée d'une valeur par défaut pour le nom d'utilisateur a donc été abandonnée pour rendre le programme plus accessible. La valeur du nom d'utilisateur serait lue de la même manière que celle du mot de passe.

\subsection{La classe Authentication}

Une nouvelle classe, \texttt{Authentication}, a été créée pour permettre la généralisation lecture du code et améliorer sa lisibilité. Celle-ci permet de créer un objet contenant un nom d'utilisateur et un mot de passe de type Optional. Ce type permet d'instancer aussi bien la valeur d'une chaîne de caractère que l'absence d'une information. Cette classe \texttt{Authentication} est lu dans une autre classe, \texttt{QueriesHelper}, qui permet de renvoyer les informations nécessaires pour se connecter à l'API. L'introduction de la classe \texttt{Authentication} permet ainsi de détecter si le nom d'utilisateur ou le mot de passe manquent et alors jeté une exception appropriée faisant échoué l'exécution du code. 

\subsection{Le projet CredsRead}
La généralisation du code permettant l'authentification nous a poussés à séparer dans un projet bien distinct de Plaquette-MIDO, le projet CredsRead. En effet, les méthodes de \texttt{QueriesHelper} ainsi que la classe \texttt{Authentication} n'avait, en grande partie, aucun lien spécifique avec plaquette-MIDO et pouvait ainsi être totalement publiées dans un autre projet pour pouvoir réutiliser plus facilement ce code dans d'autres projets. Le projet Creds-Read a ensuite été intégré au code source de Plaquette-MIDO.

\chapter[CredsRead]{CredsRead\raisebox{.3\baselineskip}{\normalsize\footnotemark}}
\footnotetext{\url{https://github.com/oliviercailloux/creds-read}}

CredsRead, pour Credentials Read, gère comme son nom l'indique la lecture des identifiants d'un utilisateur.

\section{Diagramme de classe}
Le diagramme de classe permet de avoir une idée précise du code que l'on veut écrire. Papyrus est un outil permettant de réaliser ce genre de diagramme. Son interface intuitive facilite la rédaction d'un diagramme UML(le langage standard des diagrammes de classe) lisible et rigoureux. 

\begin{figure}[!h]
    \begin{center}
    \includegraphics[width=\textwidth]{assets/doc.png}
    \end{center}
    \caption{Diagramme de classe du projet CredsRead}
\end{figure}

\chapter[Jugement délibéré, les régimes alimentaires]{Jugement délibéré, les régimes alimentaires\raisebox{.3\baselineskip}{\normalsize\footnotemark}}
\footnotetext{\url{https://github.com/barnabegeffroy/vegan_or_not}}
Ce projet a pour but de mettre en pratique les théories exposées dans l'article \textit{A formal framework for deliberated judgment}\cite{cailloux_formal_2020}. Cette article de théorie du choix s'intéresse à l'influence des arguments dans la prise de décision. Le jugement délibéré est le fait de prendre une décision en ayant pris en compte différents arguments qui ne vous feront pas changer d'avis. Pour avoir des données empiriques sur ce modèle, il a été convenu de s'intéresser aux régimes alimentaires proposés dans une cantine.

\section{Protocole de l'expérience}
L'idée est de proposer aux utilisateurs un site web sur lequel l'influence des arguments dans leur assentiment à une cantine végane ou non est suivie. Pour cela le site web proposera des vidéos de deux experts, un en faveur des régimes véganes, l'autre contre ce genre de régimes alimentaires. L'utilisateur aura accès à une bibliothèque de vidéos d'arguments des deux experts. Cette bibliothèque sera dynamique et évoluera en fonction des vidéos vues. Si l'utilisateur a vu l'argument 1 de l'expert A, il aura alors accès à la réponse de l'expert B, le contre-argument. Si ensuite, il visionne ce contre-argument, il aura accès au contre-contre-argument de l'expert A, et ainsi de suite jusqu'à ce que le débat sur cet argument soit clos par l'un des experts. Le site web doit également proposer des formulaires, en fonction des vidéos visionnées, afin de capturer la tendance de l'opinion de l'utilisateur et le moment où celui-ci émettra un jugement délibéré. Ces formulaires permettront d'étudier la puissance des arguments des deux experts et la tendance des utilisateurs dans leur prise de décision. Outre ces formulaires, le site web doit fournir des informations techniques sur le parcours de l'utilisateur, notamment le temps de visionnage des différentes vidéos.

\section{Création d'un site web}
Le site web doit répondre aux exigences du protocole : 
\begin{itemize}
    \item proposer une bibliothèque de vidéos dynamique.
    \item fournir des formulaires adaptés aux vidéos visionnées.
    \item récupérer les données de l'utilisateur sur le visionnage des vidéos.
\end{itemize}
En ce qui concerne le premier et dernier point, ils sont fortement liés. En effet, une fonction pourrait permettre de débloquer la réponse à un argument une fois que l'information "vidéo lue jusqu'à la fin" serait transmise. Pour réaliser cela, \textit{video.js}, lecteur open source, permet d'obtenir de nombreuses données nécessaires pour le projet.

Ce projet a été réalisé avec une étudiante en nutrition, il fallait alors également trouver une interface opensource permettant d'éditer un site web facilement.

\subsection{WordPress}
WordPress est un logiciel de conception web (CMS). Il est utilisé par 35\% des sites web dans le monde. WordPress permet de réaliser un site web de qualité sans avoir besoin de compétences techniques importantes (HTML, JavaScript). Il propose différents modèles adaptables facilement et plus de 30 000 extensions permettant d'accéder facilement à de multiples fonctionnalités (vidéos, formulaires, analyse de fréquentations, ...). WordPress semble ainsi être l'interface opensource parfaite pour éditer le site web facilement tout en respectant le protocole.

\begin{figure}[!ht]
    \begin{center}\includegraphics[width=\textwidth]{assets/wp.PNG}
    \end{center}
    \caption{Capture d'écran de l'interface de WordPress}
\end{figure}
Une contrainte s'ajoutait tout de même en utilisant WordPress. Le fait de dépendre d'extensions pouvant évoluer et ne plus être compatible avec nos exigeances était en effet un risque à éviter. De plus, le code source est très peu accessible sur WordPress, il est très compliqué de réaliser un site web sur mesure lorsque celui-ci exige des points techniques très précis.

Il a donc été décidé de concevoir l'esthétique du site sur WordPress (affichage, thème, polices, images). La partie technique (dynamisme, récupération des données) serait développée sur Jekyll.

\subsection{Jekyll}
\subsubsection{Fonctionnement}
Jekyll est un générateur de site statique\footnote{Une page web statique est une page web dont le contenu ne varie pas en fonction des caractéristiques de la demande.}. Ce logiciel permet de générer facilement une architecture web convaincante. Jekyll propose un système de modèle de page. Celui-ci permet d'obtenir des pages suivant les mêmes caractéristiques (styles, en-tête, pied de page, barre de navigation, ...) sans recopier sur chaque page le code HTML nécessaire. Le code permet également la conversion de fichier Markdown\footnote{Markdown est un langage de balisage offrant une syntaxe facile à lire et à écrire.}, éditable facilement, en page HTML. Chaque fichier doit er par une en-tête lue par Jekyll (voir ci-dessous). Celle-ci contient les informations permettant à Jekyll de générer une page HTML. Le fichier ne peut contenir que cette en-tête, comme ci-dessous.
\begin{figure}[!ht]
\begin{lstlisting}[language=Ruby]
---
layout: post
title:  "Le premier argmuent de l'expert A!"
date:   2020-03-27
excerpt: "S1a"
image : {{page.image}}
---
\end{lstlisting}
\caption*{Contenu d'un fichier suivant le modèle \texttt{post} pour générer une page vidéo}
\end{figure}

Jekyll va, en suivant ces quelques lignes suivrent, générer la page web (voir figure~\ref{s1a}) à partir du modèles post. L'annexe~\ref{sec:shtml} détaille le code HTML généré par Jekyll et permet de se rendre compte de l'efficacité de Jekyll, seulement sept lignes de code ont suffit à générer 190 lignes de code HTML. La page contient une vidéo qui est généré par l'\texttt{excerpt} (l.72 à 75).

\vspace{1cm}
\begin{figure}[!ht]
    \begin{center}
        \includegraphics[width=0.8\textwidth]{assets/s1a.PNG}
        \caption{Capture d'écran de la page web généré par les lignes de codes ci-dessus}
        \label{s1a}
    \end{center}
\end{figure}

\subsubsection{Thèmes}
Jekyll possède une bibliothèque de thème opensource avec des nombreux modèles préexistants. Le thème de la figure~\ref{s1a} avait d'abord été choisi en attendant de récuprer les données esthétiques du site WordPress. Cependant ce thème présentait quelques disfonctionnements et le thème de la figure a finalement été choisi

\begin{figure}[!h]
    \begin{center}
    \includegraphics[width=0.7\textwidth]{assets/newtheme.PNG}
    \end{center}
    \caption{Nouveau thème Jekyll utilisé provisoirement dans l'attente de l'esthétique WordPress}
\end{figure}

\section{Déploiement sur Internet}
\subsection{Jekyll}
Jekyll a été créé par le fondateur de GitHub, Tom Preston-Werner. Son déploiement sur Jekyll est relativement simple. Il suffit de déposer les fichiers générés par Jekyll sur un dépôt GitHub pour que ceux-ci soient déployés sur un site web, une page GitHub.

La génération des fichiers par Jekyll est automatisé de la même manière que plaquette-MIDO avec Travis-CI (voir section~\ref{sec:automatisation}). Le dépôt vers lequel les fichiers sont déployés est \href{https://github.com/barnabegeffroy/vegan_or_not}{\textcolor{blue}{\underline{vegan\_or\_not}}}. GitHub va donc créer une page GitHub à partir du contenu de ce dépôt. On obtient ainsi notre site web, \href{https://barnabegeffroy.github.io/vegan_or_not/}{\textcolor{blue}{\underline{Vegan or not vegan}}}.

\subsection{WordPress}
Le déploiement de la page WordPress sur Internet est un peu plus complexe. Pour un déploiement optimal, WordPress préconise de passer par un hébergeur web adapté et payant. Une extension WordPress existe cepandant pour convertir le contenu WordPress en statique, lisible par GitHub pour générer une page GitHub. Cette extension s'appelle WP2Static. Elle crée un répertoire contenant les fichiers statiques. Il suffit de pousser ces derniers vers un dépôt GitHub pour que le site web soit déployé. Un exemple de page WordPress déployé sur une page GitHub est disponible sur ce \href{https://barnabegeffroy.github.io/static-wp/}{\textcolor{blue}{\underline{lien}}}.

\chapter*{Remerciements}

\addcontentsline{toc}{chapter}{Remerciements}
Je remercie chaudement Olivier Cailloux qui m’a suivi tout au long de ce stage. Premièrement, il a accepté de m'encadrer malgré mon peu d'expérience en développement. Secondement, très pédagogue et à l’écoute, il a su me transmettre son goût pour l'informatique. Le partage de ses cours m'a été d'une grande aide à la compréhension des travaux à réaliser. Je le remercie également pour sa patience dans la relecture de mes codes parfois archaïques et pour nos périodes de riches échanges. Je garderai un très bon souvenir de cette période pratique.

\vspace{30pt}
{\let\clearpage\relax\chapter*{Conclusion
}}
\addcontentsline{toc}{chapter}{Conclusion}

Ce stage m'aura ainsi permis de découvrir davantage le domaine du développement informatique. À travers les différents projets menés, j'ai pu explorer différentes voies du développement : site web, projets Maven/Java, intégration continue... et de maîtriser de nombreux outils et langages informatiques : \texttt{Git}, Travis-CI, Jekyll, Maven, WordPress, HTML, JavaScript... Ce stage m’a ainsi fortement conforté dans le choix d'une formation axée sur l'informatique et le développement.



\newpage
\pagenumbering{roman}
%Ne pas numéroter cette partie

\addcontentsline{toc}{part}{Annexes}
\part*{Annexes}
\newpage
%Rajouter la ligne "Annexes" dans le sommaire

\chapter*{Codes Plaquette-MIDO}
\addcontentsline{toc}{chapter}{Codes Plaquette-MIDO}

%changer le format des sections, subsections pour apparaittre sans le num de chapitre
\makeatletter
\renewcommand{\thesection}{\@arabic\c@section}
\makeatother

%recommencer la numérotation des section à "1"
\setcounter{section}{0}

\section{Scripts pour Travis-CI}

\subsection{writeWSDL.sh}
\label{sec:writeWSDL}
\lstinputlisting[language=bash]{./assets/writeWSDL}

\vspace{1cm}

\subsection{cibuild.sh}
\label{sec:cibuild}
\lstinputlisting[language=bash]{./assets/cibuild}

\chapter*{Codes CredsRead}
\addcontentsline{toc}{chapter}{Codes CredsRead}

%changer le format des sections, subsections pour apparaittre sans le num de chapitre
\makeatletter
\renewcommand{\thesection}{\@arabic\c@section}
\makeatother

%recommencer la numérotation des section à "1"


\section{Classe CredsReader}

\subsection{Méthode readCredentials()}
\label{sec:readCredentials()}
\lstinputlisting[language=Java, firstline=166, lastline=227]{./assets/CredsRead.java}

\vspace{1cm}

\subsection{Méthode getCredentials()}
\label{sec:getCredentials()}
\lstinputlisting[language=Java, firstline=133, lastline=152]{./assets/CredsRead.java}

\chapter*{Codes Jugement Délibéré}
\addcontentsline{toc}{chapter}{Codes Jugement Délibéré}

%changer le format des sections, subsections pour apparaittre sans le num de chapitre
\makeatletter
\renewcommand{\thesection}{\@arabic\c@section}
\makeatother

%recommencer la numérotation des section à "1"


\section{Jekyll}

\subsection{Exemple de code HTML généré par Jekyll}
\label{sec:shtml}
\lstinputlisting[language=HTML]{./assets/s1a.html}

\vspace{1cm}



\newpage

%récupérer les citation avec "/footnotemark"
\nocite{*}

%choix du style de la biblio
\bibliographystyle{plain}
%inclusion de la biblio
\bibliography{bibliographie.bib}
%voir wiki pour plus d'information sur la syntaxe des entrées d'une bibliographie
\thispagestyle{empty}
\setcounter{page}{0}

\end{document}