\chapter*{Introduction}

Le développement agile est une méthodologie qui vise à délivrer rapidement une solution fonctionnelle et d'améliorer progressivement le code. Cette méthode permet une simplification du code et d'explorer des voies suggérées lors du développement et inenvisagées initiallement. L'ajout de tests fréquents permet de ne pas se perdre dès que le code renvoie une erreur. Une communication régulière entre les développeurs est également nécessaire en développement agile.

Lors de ce stage, plusieurs projets ont été réalisés en suivant cette méthode. Il seront exposés dans ce rapport. Premièrement, quelques détails techniques seront exposés. Ensuite, les quatre projets seront présentés. Le premier projet traite du développement d'une application web gérant la présence des élèves au sein d'un master de l'Université Paris-Dauphine. Le second projet s'intéresse à une application générant un fichier PDF contenant le programme du même master. Le troisième projet est un projet découlant du second. Celui-ci n'était initiallement pas prévu mais le méthodologie agile nous a conduit à développer une système d'authentification. Le dernier projet est la mise en pratique de l'article \textit{A formal framework for deliberated judgment}\cite{cailloux_formal_2020}. Une annexe à la fin du document propose certains codes utilisés dans les projets. Tous les codes sont également disponibles sur mon profil GitHub : \url{https://github.com/barnabegeffroy}.